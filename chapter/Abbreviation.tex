% !TEX root = ../main.tex
\chapter{简称列表}
  
\begin{longtable}{p{2.5cm}<{\centering}p{10.0cm}<{\centering}}
%\caption{中英文的简称列表}\centering
  \hline
  \textbf{英文简称} & \textbf{中文含义及英文全称}  \\  \hline
  \endfirsthead % 表示这是第一页的表头
  \hline
  \textbf{英文简称} & \textbf{中文含义及英文全称} \\ % 后续页表头,与第一页相同
  \hline
  \endhead
  % 表格内容部分
%\hline
%简称 &全称     \\ \hline
AI & 人工智能(Artificial intelligence)   \\ 
DL & 深度学习(Deep Learning) \\ 
AD    &阿尔茨海默症(Alzheimer's disease)    \\ 
PD    & 帕金森疾病(Parkinson's disease)  \\
CT  & 计算机断层成像(Computed Tomography)    \\
MRI    &磁共振成像(magneticrResonance imaging)    \\ 
SPECT  & 单光子发射计算机断层显像(single-photon emission computed tomograph)    \\
PET  & 正电子发射型计算机断层显像(Positron Emission Tomography)    \\ 
fMRI  & 功能性磁共振成像(functional magnetic resonance imaging)    \\
sMRI  &结构磁共振成像(structural Magnetic Resonance Imaging)    \\
X-Ray  &X射线(X Radiography) \\
Microscopy    &显微影像(Microscopic Imaging)    \\
 US   &超声成像(Ultrasound)    \\
DTI  &弥散张量(Diffusion Tensor Imaging)     \\
DAT  &多巴胺转运蛋白(Dopamine transporter)     \\
NC  & 正常认知(Normal Cognition)    \\
EMCI  & 早期轻度认知障碍(Early Mild Cognitive Impairment)    \\
LMCI  & 晚期轻度认知障碍(Late Mild Cognitive Impairment)    \\
SOTA  &最前沿的技术水平(state-of-the-art)     \\
 NSCT   &非子采样轮廓波变换(Non-Subsampled Contourlet Transform)    \\
 CNN   &  卷积神经网络(Convolutional Neural Network)  \\


\hline
%\end{tabular}
\end{longtable}
%\end{table}