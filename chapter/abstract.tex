% !TEX root = ../main.tex
\begin{abstract}
\addcontentsline{toe}{chapter}{\bfseries Abstract(In Chinese)}



%%  预答辩后更改的摘要内容(要求 <= 2000字):
医学影像数据是疾病诊断的关键信息源,传统的影像诊断手段因人为因素易产生误诊和漏诊的问题。因此,人工智能辅助诊断的需求日益迫切。对于脑部疾病,特别是阿尔茨海默病(AD)和帕金森病(PD)等神经退行性疾病,这些疾病的诊断主要依赖于CT、MRI、PET等影像检查,这些检查能提供脑部结构和功能的详细信息。
由于脑疾患类型多样且诊断流程复杂,因此整合多模态影像数据成为提升诊断精确度与全面性的关键环节。
一方面,现行的多模态影像融合策略在一定程度上忽视了医学信息与影像特征间的内在关联性,倾向于简单套用自然图像融合技术,这种局限性在脑疾病诊断中限制了其实际效能。因此,本研究旨在从耦合医学信息与影像特征出发探究多模态融合技术新方法,增强对脑疾病诊断的准确性和医学可解释性。

另一方面,,,,,,。本研究旨在通过人工智能和影像组学的综合应用,结合机器学习和深度学习技术,对脑肿瘤、AD和PD的多模态医学数据进行智能辅助诊断。本研究立足临床实际需求,紧紧围绕着基于医学影像组学的智能辅助诊断算法研究,主要开展了以下工作:


(1),,,,,,,,,,,。

(2),,,,,,,,,,,。

(3),,,,,,,,,,,,,,,。

(4),,,,,,,,,,,,,,,。

(5),,,,,,,,,,,,,,,。

本研究充分利用影像组学(多模态和异构数据),以机器学习或深度学习为基础,充分发挥人工智能在医疗领域的潜力,以辅助医生的临床决策。通过为医疗团队提供更加精准详实的数据分析及丰富的临床洞见,旨在为患者提供更个性化和有效的治疗方案,从而提升患者的生存质量、延长生存期。


\keywords{多模态; 医学影像分析; 影像组学; 深度学习; 智能辅助诊断}
\end{abstract}

\clearpage{\cleardoublepage}
\newpage

\begin{englishabstract}
\addcontentsline{toe}{chapter}{\bfseries Abstract(In English)}
Medical imaging data serves as a critical source of information for disease diagnosis, and traditional imaging diagnostic methods are prone to misdiagnosis and missed diagnosis due to human factors. As a result, the demand for artificial intelligence-assisted diagnosis is increasingly pressing. For brain diseases, particularly neurodegenerative disorders such as Alzheimer's Disease (AD) and Parkinson's Disease (PD), the diagnosis of these conditions heavily relies on imaging examinations like CT, MRI and PET, which provide detailed structural and functional information about the brain.

On the other hand, .
Grounded in actual clinical needs, this research focuses centrally on the investigation of intelligent auxiliary diagnostic algorithms based on medical radiomics, and it primarily encompasses the following tasks: 

(1) ,,,,,,,,,,

(2) ,,,,,,,,,,,,.


\englishkeywords{Multi-modal, Medical Image Analysis, Radiomics, Deep Learning, Intelligent Assisted Diagnosis}
\end{englishabstract}
