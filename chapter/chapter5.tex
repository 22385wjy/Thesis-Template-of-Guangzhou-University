% !TEX root = ../main.tex
\chapter{帕金森疾病的智能辅助预测} 
\label{chapter:pdPrediction}

PD的发生机制因素除复杂性外,还涵盖了遗传背景、环境暴露及生活习惯等多个层次的交织影响。其中,$\alpha$-突触核蛋白聚合现象是其生物学标识的核心组成部分。现有多模态影像整合技术在探究医学信息与影像特征深层次相关性方面存在不充分的问题,这在一定程度上限制了其在脑部疾病诊断中的表现力。
。。。。。。。。。。。。。

\section{引言}

\section{基于机器学习的智能辅助预测技术}\label{chapter5.1:pdPredictReview}



\section{临床信息主导的智能辅助进展预测}\label{chapter5.2:pdCMF-Net}

\subsection{智能辅助预测的数据与方案}

   
\subsubsection{进展预测的数据说明}


\subsubsection{进展预测的特征选择} 


\subsubsection{跨模态数据融合的进展预测}


\subsection{智能辅助预测的实验分析}


\subsection{进展预测模型的探索与发现}  




\section{本章小结}
本章节主要对PD的预测展开了研究与分析,包括PD的静态与动态预测。其中,
,,,,,,,。