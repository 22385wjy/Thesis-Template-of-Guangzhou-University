% !TEX root = ../main.tex
\chapter{多模态脑影像的智能辅助融合} 
\label{chapter:multiFusion}
%不同疾病依赖的影像检查类别不同,本文工作主要关注脑疾病的诊断,脑疾病种类繁多,诊断过程复杂,因此需要多模态影像技术共同作为辅助诊断依据,以提高诊断的准确性和全面性。
不同疾病依赖的影像检查类别不同,不同的影像在揭示不同脑部病变的特异性表现上具有不可替代的价值。本文的工作聚焦于脑部疾病这一多元且复杂的诊断领域,鉴于其多样化的病理类型与诊断挑战,单一影像技术往往不足以实现精确诊断。因此,核心探讨的是如何有效地整合运用多种影像模态,旨在最大程度地提升脑疾病诊断的精确度与完整性。
%随着影像处理技术的不断成熟和进步,影像融合技术正逐渐成为众多研究领域的瞩目焦点,其中医学影像融合技术的崛起更是为医疗领域带来了全新的可能性。
在这一章中,提出了两种针对多模态医学影像融合算法。一种是医学语义引导的脑影像融合双分支网络(MsgFusion)\upcite{wen2023msgfusion},另外一种是多维特征自适应线性融合网络(MdAFuse)\upcite{wen2024tip},分别于\ref{chapter3.1:MsgFusion}和\ref{chapter3.2:MdAFuse}小节详细介绍。

\section{基于脑影像的临床语义引导融合}\label{chapter3.1:MsgFusion}


\subsection{引言}
自20世纪90年代以来,影像融合技术在医学领域得到了发展和应用。然而,,,,,,,,,,,,

\subsection{相关工作}
现有的融合技术涵盖了传统的和以深度学习为基础的各类方法。在这一部分中,本节主要介绍了医学影像融合的相关工作,影像处理中的傅里叶变换和HSV颜色空间变换的理论。

\subsubsection{医学影像融合}

\subsubsection{频域变换在影像处理中的应用}

\subsubsection{彩色空间变换在影像处理中的应用}

    
\subsection{医学语义引导的双分支网络}


\subsubsection{灰度与V颜色分量结合的方案}

\subsubsection{双分支的分层融合与重构}


\subsubsection{损失函数构建及参数设置}

\subsection{多模态脑影像融合的实验}


\section{基于脑影像的多维特征自适应融合} \label{chapter3.2:MdAFuse}

\subsection{引言}
医学成像在,,,,,,,。




\subsection{相关工作}

\subsection{多维特征自适应线性融合网络}

\subsubsection{多维度特征的提取策略}\label{chapter3.2:Feature_extraction}

\subsubsection{多维度特征的融合策略}\label{chapter3.2:Fusion_strategy}


\subsubsection{损失函数的构建}\label{chapter3.2:Loss_function}

    
\subsubsection{关键特征的增强方案}\label{chapter3.2:Enhancement_of_key_features}

\subsection{多模态影像融合的实验}



\section{本章小结}
在\ref{chapter3.1:MsgFusion}节中,本节介绍了一种基于MS-Info引导的脑部疾病影像深度特征融合方法,被称为MsgFusion。该方法通过,,,,,,,。

在\ref{chapter3.2:MdAFuse}小节中,本节提出了一种新的基于DL的融合框架,被称为MdAFuse,,,,,,,,,。

未来,计划扩展框架,将两种以上的医学影像进行融合,并应用于更广泛的临床诊断场景。另外,将更深入地探讨如何将神经科学与人工智能方法更加紧密地结合,以实现更全面和准确的医学影像融合。