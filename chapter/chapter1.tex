\chapter{绪论} 
\label{chapter:Introduction}
自从影像组学这一概念被首次提出以来\upcite{lambin2012radiomics},基于医学影像组学的智能辅助诊断技术便持续成为计算机应用研究领域的研究热点\upcite{lambin2012radiomics,lo2019computer}。在特定情况下,它能够为医生提供有效的决策支持,提高诊断准确性。随着数字化医学影像的广泛应用,这一技术在实际临床诊断中的需求和可控性大大增强,为深度学习在医学影像领域的落地提供了重要的突破口。这标志着智能辅助诊断技术将进一步与先进的计算机科学、机器学习相结合,为医学诊断领域带来更为准确、高效的解决方案,推动医学科技取得新的突破。

\section{研究背景及意义}


    


\section{国内外研究现状}
随着科技的飞速发展,基于医学影像组学的智能辅助诊断技术正引领着医学领域的革新浪潮。在这一领域的研究,深入挖掘了传统方法与深度学习方法两大方向,呈现出多层次、多维度的探索。传统医学影像组学诊断依赖于人工设计特征和构建分类器,虽然为医生提供了一定的支持,但其在面对复杂疾病或多样化影像时表现出的局限性逐渐凸显。然而,近年来深度学习的兴起,特别是卷积神经网络的引入,赋予了模型自主学习特征的能力,显著提升了诊断性能。智能辅助诊断方面,深度学习模型的广泛应用为研究者提供了前所未有的研究机会和成果。

  \begin{figure}[htbp]
      \centering  
      \includegraphics[width=0.9\linewidth]{figs/paperNumber.pdf}  
      \caption{按年搜索PubMed上三个关键词的发文量统计}\label{paperNumber}
    \end{figure}



\subsection{多模态医学影像融合}


\section{本文的主要工作}
本项研究聚焦于智能辅助影像组学分析技术在医学辅助诊断领域中的应用,特别关注脑肿瘤、AD和PD。关键技术手段涵盖影像融合、影像分类与疾病预测。本文的主要工作包括:
\begin{itemize}
    \item 在多模态医学影像融合方面,,,,,,,,,,,,,。
    \item 在AD诊断方面的研究上,,,,,,,,,,,,,。
    \item 在PD诊断领域,本文,,,,,,,,,,,,,,,,,,,,,,,,,。
\end{itemize}

   


%\section{本文的篇章结构}
全文共分为六个章节,整体的篇章结构如图\ref{wholeWorks}所示。
   
第一章绪论。。。。。。。。。。。。。。。。。。。。
    

第二章基础理论。。。。。。。。。。。。。。。。。


第三章聚焦于多模态影像融合。。。。。。。。。。。。。。。。

第四章关注AD影像分类。。。。。。。。。。。。。。。。。。


第五章探讨PD疾病预测。。。。。。。。。。。。。。。。。

第六章回顾总结了全文研究内容,阐述了在当前研究工作中所存在的不足,并展望了未来可能的研究方向和工作设想,为基于医学影像组学的智能诊断技术发展提供了实践经验和研究思路。