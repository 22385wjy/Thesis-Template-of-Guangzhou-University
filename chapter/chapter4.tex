% !TEX root = ../main.tex
\chapter{阿尔茨海默症的智能辅助分类} 
\label{chapter:fineClassify}
当前神经网络驱动的图像分类算法普遍存在仅能捕获局部空间特征的问题,这在一定程度上制约了特征抽取的有效性,进而可能影响到分类性能的整体准确性。特别是在AD早期阶段的诊断分类任务中,面对从EMCI、MCI到LMCI这一连续进展阶段的精细化分类任务,亟需发展更加稳健且智能的分类策略,以实现对疾病进程的精准评估与预测。
在这一章中,将介绍一种以影像为导向的新小波卷积单元神经网络。一方面,,,,,,,,,,,

\section{引言}
AD是在老年人群中非常常见的一种疾病。目前,还没有有效的治疗方法可以治愈AD或改变其进展。MCI是介于AD和NC之间的中间阶段。临床研究表明,MCI可分为EMCI和LMCI。EMCI阶段是可逆的,及时发现和干预可以避免AD的发展,而在LMCI阶段及时诊断和治疗可以延缓AD的发展或治愈AD,因此痴呆的早期发现和诊断将成为主要目标。AD的早期准确诊断是一个有意义和挑战性的任务。





\section{相关工作}\label{chapter4.2}

\subsection{基于传统智能辅助分类技术}



\subsection{基于新型智能辅助分类技术}



\subsection{基于频域理论分析的技术} 



\section{细粒度多分类的辅助诊断网络}\label{chapter4:WCU-net}
在细粒度和多分类任务中,本章提出了一种新的AD分类方法。接下来,将介绍本章自定义的小波卷积单元 (Wavelet Convolution Unit,WCU)、细粒度多分类网络(WCU-Net),以及WCU-Net对AD的细粒度多分类问题,分别对应\ref{paper3WCU}节、\ref{paper3WCU-Net}节与\ref{paper3WCUNetClassify}节。

\subsection{小波卷积单元的设计思路}\label{paper3WCU}


\subsection{细粒度多分类的网络结构}\label{paper3WCU-Net}

\subsection{临床特征驱动型辅助分类}\label{paper3WCUNetClassify}

\subsubsection{临床特征驱动的脑影像数据}


\subsubsection{细粒度与多分类的技术路线}\label{4.3.3.2}

\section{细粒度多分类的实验}\label{chapter4.4}



\section{本章小结}\label{chapter4.5}
本章提出了一种新的网络WCU-Net,它,,,,,。