% !TEX root = ../main.tex
\chapter{理论与技术基础} 
\label{chapter:Basis}
因不同的医学影像来源于不同的成像原理,本章深入分析了七种医学影像成像技术,包括CT、MRI、X射线等,以满足医学领域辅助诊断的需求。其次,探讨了智能辅助诊断技术理论基础,如传统机器学习方法(SVM)和深度学习方法(CNN)在医学影像分类中的应用,它们的结合是提高医学诊断准确性和效率的前沿趋势。并研究了多种层面的影像融合策略,包括像素级、特征级和决策级的方法,为实际应用提供了启示。最后,引入了较全面的评估方法,包括主观和客观评价,用于评估多模态影像融合与分类算法的性能。

\section{医学影像的成像技术原理}
在医学领域,,,,,,,,,,,。不同多模态医学影像具有不同的表现形式,分别如图\ref{mul_imaging}所示。

   \begin{figure*}[htbp]
      \centering  
      \includegraphics[width=0.9\linewidth]{figs/mul_imaging.png}
      \caption{七种不同类型的医学影像}\label{mul_imaging}
    \end{figure*}



\section{智能影像组学技术的性能评估}



\section{本章小结}
在这一章节中,本章以,,,,。

另外,本章进一步深入研究了,,,,,,,,。