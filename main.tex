%% TeX mode: XeLaTeX + bibTeX + XeLaTeX + XeLaTeX
%% Thesis Template of Guangzhou University

\PassOptionsToPackage{quiet}{fontspec} % 消除字体警告
\expandafter\def\csname CTEX@spaceChar\endcsname{\hspace{1em}}
\documentclass[ notypeinfo]{style/GZHUthesis}

% --- 页面设置:上下左右各 25mm ---
\usepackage{geometry}
\geometry{a4paper, margin=25mm, footskip=10mm} 
\setlength{\headwidth}{\textwidth} % 强制页眉宽度等于文本宽度
\makeatletter \makeatother         % 刷新设置

% ---  标题设置:单倍行距,段前段后 0.5 行 ---
\ctexset{
    chapter={
        format={\bfseries\heiti\zihao{-2}\singlespacing}, 
        beforeskip={0.5 \baselineskip},
        afterskip={0.5 \baselineskip},
        number={\arabic{chapter}}
    },
    section={
        format={\bfseries\heiti\zihao{3}\singlespacing\raggedright},
        beforeskip={0.5 \baselineskip},
        afterskip={0.5 \baselineskip}
    },
    subsection={
        format={\bfseries\heiti\zihao{-3}\singlespacing},
        beforeskip={0.5 \baselineskip},
        afterskip={0.5 \baselineskip}
    }
}

%盲审模式,正常模式请注释掉
%\blindtrue
\blindfalse

%默认博士论文, 如需要硕士论文, 注释掉下行
\phdtrue
%\phdfalse

%字体设置
\setsansfont{Arial}

%\citeyearn{XXXX} 自动在\citeyear后面加上“年”
\hypersetup{hidelinks} %取消链接的颜色

%% 为了方便输入特殊符号, 添加如下新命令:
\medmuskip=2mu        %水平间距调整: 二元运算符 "+, -, <"
\thickmuskip=3mu      %水平间距调整: 关系符号调整 "="


\graphicspath{{chapter/}{figures/}}  % 设置图形文件的搜索路径
%\ctexset{section={format+={\flushleft}}}  % 小节标题靠左对齐

% 添加新文件时, 需要在此处加入文件名, 方便控制编译结果
\ifblind
\includeonly{  %% 选择盲审时要编译的章节
    chapter/abstract,
    chapter/chapter1,
    chapter/chapter2,
    chapter/chapter3,
    chapter/chapter4,
    chapter/chapter5,
    chapter/conclusion,
}
\else
\includeonly{  %% 选择要编译的章节, 注释掉的章节不会编译, 平时可注释掉某些章节来提升整体编译速度
    chapter/abstract,
    chapter/chapter1,
    chapter/chapter2,
    chapter/chapter3,
    chapter/chapter4,
    chapter/chapter5,
    chapter/conclusion,
    chapter/Abbreviation,
    chapter/publications,
    chapter/projects,
    chapter/thanks
}
\fi


% 自定义命令
\include{chapter/defcommands}

% 论文基本信息设置(包含作者, 论文标题等)
\include{chapter/cover.cfg} 

\pgfplotsset{compat=1.18}

\makeatletter
\let\c@lofdepth\relax
\let\c@lotdepth\relax
\makeatother
\usepackage{tocloft} % 设置目录中的条目间距
\renewcommand{\cftchapleader}{\cftdotfill{\cftdotsep}} % 设置目录格式 接着添加:
\renewcommand\cftchapfont{\songti \zihao{4}} %chapter字号字体

\begin{document}
    \begin{sloppypar} % 处理行溢出
        \let\standardtilde=\relax 
        % 此命令可将其后所有"~"改为不可换行的空格符, 因为ctex宏包重新定义"~"是可换行的空格符
        \standardtilde
        \CJKspace     
        
        %\makechinesetitle  % 中文封面
        \makedefendpage    % 中文答辩页
        
        
        \frontmatter  % 前言部分
        \pagenumbering{Roman} % 页码大写罗马字体
        % !TEX root = ../main.tex
\begin{abstract}
\addcontentsline{toe}{chapter}{\bfseries Abstract(In Chinese)}



%%  预答辩后更改的摘要内容(要求 <= 2000字):
医学影像数据是疾病诊断的关键信息源,传统的影像诊断手段因人为因素易产生误诊和漏诊的问题。因此,人工智能辅助诊断的需求日益迫切。对于脑部疾病,特别是阿尔茨海默病(AD)和帕金森病(PD)等神经退行性疾病,这些疾病的诊断主要依赖于CT、MRI、PET等影像检查,这些检查能提供脑部结构和功能的详细信息。
由于脑疾患类型多样且诊断流程复杂,因此整合多模态影像数据成为提升诊断精确度与全面性的关键环节。
一方面,现行的多模态影像融合策略在一定程度上忽视了医学信息与影像特征间的内在关联性,倾向于简单套用自然图像融合技术,这种局限性在脑疾病诊断中限制了其实际效能。因此,本研究旨在从耦合医学信息与影像特征出发探究多模态融合技术新方法,增强对脑疾病诊断的准确性和医学可解释性。

另一方面,,,,,,。本研究旨在通过人工智能和影像组学的综合应用,结合机器学习和深度学习技术,对脑肿瘤、AD和PD的多模态医学数据进行智能辅助诊断。本研究立足临床实际需求,紧紧围绕着基于医学影像组学的智能辅助诊断算法研究,主要开展了以下工作:


(1),,,,,,,,,,,。

(2),,,,,,,,,,,。

(3),,,,,,,,,,,,,,,。

(4),,,,,,,,,,,,,,,。

(5),,,,,,,,,,,,,,,。

本研究充分利用影像组学(多模态和异构数据),以机器学习或深度学习为基础,充分发挥人工智能在医疗领域的潜力,以辅助医生的临床决策。通过为医疗团队提供更加精准详实的数据分析及丰富的临床洞见,旨在为患者提供更个性化和有效的治疗方案,从而提升患者的生存质量、延长生存期。


\keywords{多模态; 医学影像分析; 影像组学; 深度学习; 智能辅助诊断}
\end{abstract}

\clearpage{\cleardoublepage}
\newpage

\begin{englishabstract}
\addcontentsline{toe}{chapter}{\bfseries Abstract(In English)}
Medical imaging data serves as a critical source of information for disease diagnosis, and traditional imaging diagnostic methods are prone to misdiagnosis and missed diagnosis due to human factors. As a result, the demand for artificial intelligence-assisted diagnosis is increasingly pressing. For brain diseases, particularly neurodegenerative disorders such as Alzheimer's Disease (AD) and Parkinson's Disease (PD), the diagnosis of these conditions heavily relies on imaging examinations like CT, MRI and PET, which provide detailed structural and functional information about the brain.

On the other hand, .
Grounded in actual clinical needs, this research focuses centrally on the investigation of intelligent auxiliary diagnostic algorithms based on medical radiomics, and it primarily encompasses the following tasks: 

(1) ,,,,,,,,,,

(2) ,,,,,,,,,,,,.


\englishkeywords{Multi-modal, Medical Image Analysis, Radiomics, Deep Learning, Intelligent Assisted Diagnosis}
\end{englishabstract}
  % 摘要
        \begin{spacing}{1.0}
            \setlength\cftbeforechapskip{6pt}
            \setlength\cftbeforesecskip{6pt}
            \setlength\cftbeforesubsecskip{6pt}
            %\tableofcontents             % 目录
            %%让目录居中
            \begin{center}
                \tableofcontents
            \end{center}  
        \end{spacing}
        %\mainmatter   %% 正文部分
        % ---  正文部分:1.5 倍行距,段前后无空行 ---
        \mainmatter
        \onehalfspacing           % 开启全局 1.5 倍行距
        \setlength{\parskip}{0pt} % 段前段后无空行
        
        
        % 添加新章节时必须在此处声明
        \chapter{绪论} 
\label{chapter:Introduction}
自从影像组学这一概念被首次提出以来\upcite{lambin2012radiomics},基于医学影像组学的智能辅助诊断技术便持续成为计算机应用研究领域的研究热点\upcite{lambin2012radiomics,lo2019computer}。在特定情况下,它能够为医生提供有效的决策支持,提高诊断准确性。随着数字化医学影像的广泛应用,这一技术在实际临床诊断中的需求和可控性大大增强,为深度学习在医学影像领域的落地提供了重要的突破口。这标志着智能辅助诊断技术将进一步与先进的计算机科学、机器学习相结合,为医学诊断领域带来更为准确、高效的解决方案,推动医学科技取得新的突破。

\section{研究背景及意义}


    


\section{国内外研究现状}
随着科技的飞速发展,基于医学影像组学的智能辅助诊断技术正引领着医学领域的革新浪潮。在这一领域的研究,深入挖掘了传统方法与深度学习方法两大方向,呈现出多层次、多维度的探索。传统医学影像组学诊断依赖于人工设计特征和构建分类器,虽然为医生提供了一定的支持,但其在面对复杂疾病或多样化影像时表现出的局限性逐渐凸显。然而,近年来深度学习的兴起,特别是卷积神经网络的引入,赋予了模型自主学习特征的能力,显著提升了诊断性能。智能辅助诊断方面,深度学习模型的广泛应用为研究者提供了前所未有的研究机会和成果。

  \begin{figure}[htbp]
      \centering  
      \includegraphics[width=0.9\linewidth]{figs/paperNumber.pdf}  
      \caption{按年搜索PubMed上三个关键词的发文量统计}\label{paperNumber}
    \end{figure}



\subsection{多模态医学影像融合}


\section{本文的主要工作}
本项研究聚焦于智能辅助影像组学分析技术在医学辅助诊断领域中的应用,特别关注脑肿瘤、AD和PD。关键技术手段涵盖影像融合、影像分类与疾病预测。本文的主要工作包括:
\begin{itemize}
    \item 在多模态医学影像融合方面,,,,,,,,,,,,,。
    \item 在AD诊断方面的研究上,,,,,,,,,,,,,。
    \item 在PD诊断领域,本文,,,,,,,,,,,,,,,,,,,,,,,,,。
\end{itemize}

   


%\section{本文的篇章结构}
全文共分为六个章节,整体的篇章结构如图\ref{wholeWorks}所示。
   
第一章绪论。。。。。。。。。。。。。。。。。。。。
    

第二章基础理论。。。。。。。。。。。。。。。。。


第三章聚焦于多模态影像融合。。。。。。。。。。。。。。。。

第四章关注AD影像分类。。。。。。。。。。。。。。。。。。


第五章探讨PD疾病预测。。。。。。。。。。。。。。。。。

第六章回顾总结了全文研究内容,阐述了在当前研究工作中所存在的不足,并展望了未来可能的研究方向和工作设想,为基于医学影像组学的智能诊断技术发展提供了实践经验和研究思路。
        % !TEX root = ../main.tex
\chapter{理论与技术基础} 
\label{chapter:Basis}
因不同的医学影像来源于不同的成像原理,本章深入分析了七种医学影像成像技术,包括CT、MRI、X射线等,以满足医学领域辅助诊断的需求。其次,探讨了智能辅助诊断技术理论基础,如传统机器学习方法(SVM)和深度学习方法(CNN)在医学影像分类中的应用,它们的结合是提高医学诊断准确性和效率的前沿趋势。并研究了多种层面的影像融合策略,包括像素级、特征级和决策级的方法,为实际应用提供了启示。最后,引入了较全面的评估方法,包括主观和客观评价,用于评估多模态影像融合与分类算法的性能。

\section{医学影像的成像技术原理}
在医学领域,,,,,,,,,,,。不同多模态医学影像具有不同的表现形式,分别如图\ref{mul_imaging}所示。

   \begin{figure*}[htbp]
      \centering  
      \includegraphics[width=0.9\linewidth]{figs/mul_imaging.png}
      \caption{七种不同类型的医学影像}\label{mul_imaging}
    \end{figure*}



\section{智能影像组学技术的性能评估}



\section{本章小结}
在这一章节中,本章以,,,,。

另外,本章进一步深入研究了,,,,,,,,。
        % !TEX root = ../main.tex
\chapter{多模态脑影像的智能辅助融合} 
\label{chapter:multiFusion}
%不同疾病依赖的影像检查类别不同,本文工作主要关注脑疾病的诊断,脑疾病种类繁多,诊断过程复杂,因此需要多模态影像技术共同作为辅助诊断依据,以提高诊断的准确性和全面性。
不同疾病依赖的影像检查类别不同,不同的影像在揭示不同脑部病变的特异性表现上具有不可替代的价值。本文的工作聚焦于脑部疾病这一多元且复杂的诊断领域,鉴于其多样化的病理类型与诊断挑战,单一影像技术往往不足以实现精确诊断。因此,核心探讨的是如何有效地整合运用多种影像模态,旨在最大程度地提升脑疾病诊断的精确度与完整性。
%随着影像处理技术的不断成熟和进步,影像融合技术正逐渐成为众多研究领域的瞩目焦点,其中医学影像融合技术的崛起更是为医疗领域带来了全新的可能性。
在这一章中,提出了两种针对多模态医学影像融合算法。一种是医学语义引导的脑影像融合双分支网络(MsgFusion)\upcite{wen2023msgfusion},另外一种是多维特征自适应线性融合网络(MdAFuse)\upcite{wen2024tip},分别于\ref{chapter3.1:MsgFusion}和\ref{chapter3.2:MdAFuse}小节详细介绍。

\section{基于脑影像的临床语义引导融合}\label{chapter3.1:MsgFusion}


\subsection{引言}
自20世纪90年代以来,影像融合技术在医学领域得到了发展和应用。然而,,,,,,,,,,,,

\subsection{相关工作}
现有的融合技术涵盖了传统的和以深度学习为基础的各类方法。在这一部分中,本节主要介绍了医学影像融合的相关工作,影像处理中的傅里叶变换和HSV颜色空间变换的理论。

\subsubsection{医学影像融合}

\subsubsection{频域变换在影像处理中的应用}

\subsubsection{彩色空间变换在影像处理中的应用}

    
\subsection{医学语义引导的双分支网络}


\subsubsection{灰度与V颜色分量结合的方案}

\subsubsection{双分支的分层融合与重构}


\subsubsection{损失函数构建及参数设置}

\subsection{多模态脑影像融合的实验}


\section{基于脑影像的多维特征自适应融合} \label{chapter3.2:MdAFuse}

\subsection{引言}
医学成像在,,,,,,,。




\subsection{相关工作}

\subsection{多维特征自适应线性融合网络}

\subsubsection{多维度特征的提取策略}\label{chapter3.2:Feature_extraction}

\subsubsection{多维度特征的融合策略}\label{chapter3.2:Fusion_strategy}


\subsubsection{损失函数的构建}\label{chapter3.2:Loss_function}

    
\subsubsection{关键特征的增强方案}\label{chapter3.2:Enhancement_of_key_features}

\subsection{多模态影像融合的实验}



\section{本章小结}
在\ref{chapter3.1:MsgFusion}节中,本节介绍了一种基于MS-Info引导的脑部疾病影像深度特征融合方法,被称为MsgFusion。该方法通过,,,,,,,。

在\ref{chapter3.2:MdAFuse}小节中,本节提出了一种新的基于DL的融合框架,被称为MdAFuse,,,,,,,,,。

未来,计划扩展框架,将两种以上的医学影像进行融合,并应用于更广泛的临床诊断场景。另外,将更深入地探讨如何将神经科学与人工智能方法更加紧密地结合,以实现更全面和准确的医学影像融合。
        % !TEX root = ../main.tex
\chapter{阿尔茨海默症的智能辅助分类} 
\label{chapter:fineClassify}
当前神经网络驱动的图像分类算法普遍存在仅能捕获局部空间特征的问题,这在一定程度上制约了特征抽取的有效性,进而可能影响到分类性能的整体准确性。特别是在AD早期阶段的诊断分类任务中,面对从EMCI、MCI到LMCI这一连续进展阶段的精细化分类任务,亟需发展更加稳健且智能的分类策略,以实现对疾病进程的精准评估与预测。
在这一章中,将介绍一种以影像为导向的新小波卷积单元神经网络。一方面,,,,,,,,,,,

\section{引言}
AD是在老年人群中非常常见的一种疾病。目前,还没有有效的治疗方法可以治愈AD或改变其进展。MCI是介于AD和NC之间的中间阶段。临床研究表明,MCI可分为EMCI和LMCI。EMCI阶段是可逆的,及时发现和干预可以避免AD的发展,而在LMCI阶段及时诊断和治疗可以延缓AD的发展或治愈AD,因此痴呆的早期发现和诊断将成为主要目标。AD的早期准确诊断是一个有意义和挑战性的任务。





\section{相关工作}\label{chapter4.2}

\subsection{基于传统智能辅助分类技术}



\subsection{基于新型智能辅助分类技术}



\subsection{基于频域理论分析的技术} 



\section{细粒度多分类的辅助诊断网络}\label{chapter4:WCU-net}
在细粒度和多分类任务中,本章提出了一种新的AD分类方法。接下来,将介绍本章自定义的小波卷积单元 (Wavelet Convolution Unit,WCU)、细粒度多分类网络(WCU-Net),以及WCU-Net对AD的细粒度多分类问题,分别对应\ref{paper3WCU}节、\ref{paper3WCU-Net}节与\ref{paper3WCUNetClassify}节。

\subsection{小波卷积单元的设计思路}\label{paper3WCU}


\subsection{细粒度多分类的网络结构}\label{paper3WCU-Net}

\subsection{临床特征驱动型辅助分类}\label{paper3WCUNetClassify}

\subsubsection{临床特征驱动的脑影像数据}


\subsubsection{细粒度与多分类的技术路线}\label{4.3.3.2}

\section{细粒度多分类的实验}\label{chapter4.4}



\section{本章小结}\label{chapter4.5}
本章提出了一种新的网络WCU-Net,它,,,,,。
        % !TEX root = ../main.tex
\chapter{帕金森疾病的智能辅助预测} 
\label{chapter:pdPrediction}

PD的发生机制因素除复杂性外,还涵盖了遗传背景、环境暴露及生活习惯等多个层次的交织影响。其中,$\alpha$-突触核蛋白聚合现象是其生物学标识的核心组成部分。现有多模态影像整合技术在探究医学信息与影像特征深层次相关性方面存在不充分的问题,这在一定程度上限制了其在脑部疾病诊断中的表现力。
。。。。。。。。。。。。。

\section{引言}

\section{基于机器学习的智能辅助预测技术}\label{chapter5.1:pdPredictReview}



\section{临床信息主导的智能辅助进展预测}\label{chapter5.2:pdCMF-Net}

\subsection{智能辅助预测的数据与方案}

   
\subsubsection{进展预测的数据说明}


\subsubsection{进展预测的特征选择} 


\subsubsection{跨模态数据融合的进展预测}


\subsection{智能辅助预测的实验分析}


\subsection{进展预测模型的探索与发现}  




\section{本章小结}
本章节主要对PD的预测展开了研究与分析,包括PD的静态与动态预测。其中,
,,,,,,,。
        % !TEX root = ../main.tex
\chapter{总结与展望} \label{chapter:Conclusions}
人工智能的飞速进步为人类探索生命奥秘开辟了广阔天地,尤其在处理生老病死等根本问题时,人们对健康和长寿的追求从未停止。在医疗领域,人工智能的广泛应用正引领着一场深刻的变革:它不仅在临床诊断中协助医生和放射科专家提高工作效率、降低误诊概率,同时让患者得以获取更多的医学信息,增强了对自身病情的理解,显著提升了他们的健康意识。

。。。。。。。。。。。。。。。。。。。。。

当前研究尚处于初级阶段,未来有望探索以下几个研究方向:

。。。。。。。。。。。。。。。。。。。。。。。。。。
        \appendix  % 附录
        \backmatter  %% 附件部分
        
        \bibliographystyle{style/gbt7714-2005-numerical}
        \bibliography{reference/myref}
        
        \ifblind
        \else
        % !TEX root = ../main.tex
\chapter{简称列表}
  
\begin{longtable}{p{2.5cm}<{\centering}p{10.0cm}<{\centering}}
%\caption{中英文的简称列表}\centering
  \hline
  \textbf{英文简称} & \textbf{中文含义及英文全称}  \\  \hline
  \endfirsthead % 表示这是第一页的表头
  \hline
  \textbf{英文简称} & \textbf{中文含义及英文全称} \\ % 后续页表头,与第一页相同
  \hline
  \endhead
  % 表格内容部分
%\hline
%简称 &全称     \\ \hline
AI & 人工智能(Artificial intelligence)   \\ 
DL & 深度学习(Deep Learning) \\ 
AD    &阿尔茨海默症(Alzheimer's disease)    \\ 
PD    & 帕金森疾病(Parkinson's disease)  \\
CT  & 计算机断层成像(Computed Tomography)    \\
MRI    &磁共振成像(magneticrResonance imaging)    \\ 
SPECT  & 单光子发射计算机断层显像(single-photon emission computed tomograph)    \\
PET  & 正电子发射型计算机断层显像(Positron Emission Tomography)    \\ 
fMRI  & 功能性磁共振成像(functional magnetic resonance imaging)    \\
sMRI  &结构磁共振成像(structural Magnetic Resonance Imaging)    \\
X-Ray  &X射线(X Radiography) \\
Microscopy    &显微影像(Microscopic Imaging)    \\
 US   &超声成像(Ultrasound)    \\
DTI  &弥散张量(Diffusion Tensor Imaging)     \\
DAT  &多巴胺转运蛋白(Dopamine transporter)     \\
NC  & 正常认知(Normal Cognition)    \\
EMCI  & 早期轻度认知障碍(Early Mild Cognitive Impairment)    \\
LMCI  & 晚期轻度认知障碍(Late Mild Cognitive Impairment)    \\
SOTA  &最前沿的技术水平(state-of-the-art)     \\
 NSCT   &非子采样轮廓波变换(Non-Subsampled Contourlet Transform)    \\
 CNN   &  卷积神经网络(Convolutional Neural Network)  \\


\hline
%\end{tabular}
\end{longtable}
%\end{table}  % 简称列表
        % !TEX root = ../main.tex
\begin{publications}{99}
\ifphd
\addcontentsline{toe}{chapter}{\bf Papers published in the period of doctor education}
\else
\addcontentsline{toe}{chapter}{\bf Papers published in the period of master education}
\fi
%\thispagestyle{empty}

%%盲审时候提交版本
\if 0
\item[1.]
**. ------------ [J]. \textit{IEEE Transactions on Multimedia (TMM)}.  (第一作者, Q1, 中科院一区, 影响因子:7.3,已发表)

\item[2.]
**. ------------ [J]. \textit{IEEE Transactions on Biomedical Engineering (T-BME)}.  (第一作者, Q1,中科院二区, 影响因子:4.6,已发表)

\item[3.]
**. ------------ [J]. \textit{IEEE Transactions on Image Processing (TIP)} (第一作者, Q1, 中科院一区, 影响因子:10.6,已在线发表)

\item[4.]
**. ------------ [J].\textit{npj Parkinsons Disease}. (第一作者, Q1, 中科院一区, 影响因子:8.7,审稿中)

\item[5.]
**. ------------ [J].\textit{图学学报}. (第一作者, 中文核心,已发表)


\item[6.]
**. ------------ [J]. \textit{放射学实践}. (第二作者,中文核心, 已发表)

\fi



%%正式答辩后提交版本
%\if 0
\item[1.]
\textbf{Jinyu Wen}, Asad Khan, Amei Chen, Weilong Peng, Meie Fang, Ping Li and C. L. Philip Chen. High-Quality Fusion and Visualization for MR-PET Brain Tumor Images via Multi-Dimensional Features[J]. \textit{IEEE Transactions on Image Processing}, May 2024, published online, DOI: 10.1109/TIP.2024.3404660. (中科院一区,IF:10.6)

\item[2.]
\textbf{Jinyu Wen}, , , , , , , ,




\end{publications}
     % 发表文章目录
        % !TEX root = ../main.tex
\begin{projects}{99}
\ifphd
\addcontentsline{toe}{chapter}{\bf Projects Participated in the period of doctor education}
\else
\addcontentsline{toe}{chapter}{\bf Projects Participated in the period of master education}
\fi
%\thispagestyle{empty}

\item[1.]
主持****研究生“基础创新”项目《 ------------ 》(项目编号: ****** )

\item[2.]
参与国家自然科学基金项目《 ------------ 》(项目号: ****** )

\item[3.]
参与**市重点研究计划项目《 ------------ 》(项目号: ****** )

\end{projects}
 % 参与项目目录
        % !TEX root = ../main.tex
\begin{thanks}
\addcontentsline{toe}{chapter}{\bf Acknowledgement}
%\thispagestyle{empty}

白驹过隙,时光匆匆。行文至此, 多年的学生时代即将谢幕,回首过去,仿佛一场梦。四年的博士生活在紧张而充实的科研中默默度过,回顾这段难忘的时光,心头留下了痛苦的泪水,也回荡着欢乐的笑声,然而更多的是对经历困难后那抹明媚阳光的深刻感受。

,,,,,,,,,,,,

最后, 感谢在百忙之中参与评阅本文和学位答辩的各位专家和老师们! 




\vskip 26pt
\hspace{10.5cm} \textit{温金玉}

\vskip 6pt
\hspace{9.0cm} 2024年5月于广州大学 
\end{thanks}
  % 致谢 
        \fi
        
    \end{sloppypar} % 处理行溢出
\end{document}
